
\documentclass[a4paper, 11pt]{article}
\usepackage{graphicx}
\usepackage{amsmath}
\usepackage[pdftex]{hyperref}

% Lengths and indenting
\setlength{\textwidth}{16.5cm}
\setlength{\marginparwidth}{1.5cm}
\setlength{\parindent}{0cm}
\setlength{\parskip}{0.15cm}
\setlength{\textheight}{22cm}
\setlength{\oddsidemargin}{0cm}
\setlength{\evensidemargin}{\oddsidemargin}
\setlength{\topmargin}{0cm}
\setlength{\headheight}{0cm}
\setlength{\headsep}{0cm}

\renewcommand{\familydefault}{\sfdefault}

\title{Data Mining: Learning From Large Data Sets\\\vspace{2mm}\Large{Fall Semester 2015}}
\author{\{usamuel, adavidso, kurmannn\}@student.ethz.ch}
\date{\today}

\begin{document}
\maketitle
\thispagestyle{empty}
\pagestyle{empty}

\section*{Large Scale Image Classification} 

% \texttt{Briefly describe the steps used to produce the solution. Feel free to add plots or screenshots if you think it's necessary. The report should contain a maximum of 2 pages.
% Keep in mind that you only have to submit one report per group. Please indicate the contribution of each group member to the project.}

\paragraph{Problem formulation.\!\!\!}
The goal of this project was to construct an SVM that classifies images into one of two classes, which are images of people versus images of nature. The images were provided in the form of precomputed feature vectors, each containing $400$ features.

\paragraph{Approach and Results.\!\!\!}
In order to run stochastic gradient descent (SGD), we chose a data-parallel approach, by having each mapper run SGD on a subset of the training data. 
	Each mapper reads values $(y_t,x_t)$, for $1\leq t\leq T$, and then computes $w_t$ using the SGD algorithm, as presented in class. The result that the mapper passes to the reducer is the average over 
all its $w_t$. The reducer then collects and averages over these results, and outputs the final classifier $w$.

We tried several approaches to compute a good classifier for the image classification problem.

At first, we used online convex programming (OCP) to compute a linear classifier. The features were used ``as is", with only the addition of a constant feature $1$ to account for an intercept. We chose $\eta_t=1/\sqrt{t}$ and $\lambda=1$. This approach yielded a score of about $60$ percent.

To improve on this, we tried to process the feature vectors in random order, as is done in SGD. We adapted the mapper to first store the entire input in a matrix, then pick elements uniformly at random. To make the most out of the available computation time, we kept on reading from the matrix randomly for $4.5$ minutes. In addition, we chose $\lambda=1/N$, with $N$ being the number of feature vectors passed to the mapper, and $\eta= 1/(\lambda t)$. 
This approach yielded a score of approximately 70 percent.

We were not able to improve on this and therefore assumed that the feature vectors were not sufficiently
linearly separable. We experimented with several non-linear feature transformations, and were able to boost our score to 81 percent with the following transformation, which uses built-in NumPy functions:
\begin{equation*}
\texttt{x <- [numpy.sqrt(x), numpy.diff(x), numpy.gradient(x)]}
\end{equation*}
This might seem like an odd transformation, but the motivation for doing this was that the original feature vectors were sparse, and so we wanted to remove some of this sparsity, which is why we use \texttt{diff} and \texttt{gradient}.

Finally, we were able to beat the hard baseline by using random Fourier features on top of the above transformed features. Sampling 3,500 of these resulted in a score of 84 percent. At this point in time, we noticed that the first entry in every original feature vector in the training set was zero, so we dropped this feature.

\paragraph{Workload distribution.\!\!\!}
Nico and Alexander wrote the initial structure for the mapper and reducer. Samuel implemented the basic OCP algorithm. Alexander extended this to include SGD and he also came up with the non-linear transformations. Samuel implemented random Fourier features (with errors), and Alexander corrected the errors, which then resulted in our final mapper. Everybody contributed to the report.

\end{document}
